\documentclass[a4paper,12pt]{article} 
\usepackage[T2A]{fontenc}			
\usepackage[utf8]{inputenc}			
\usepackage[english,russian]{babel}	
\usepackage{amsmath,amsfonts,amssymb,amsthm,mathrsfs,mathtools} 
\usepackage{cancel}
\usepackage{hhline}
\usepackage{multirow}
\usepackage[colorlinks, linkcolor = purple, citecolor = purple]{hyperref}
\usepackage{upgreek}\usepackage[left=2cm,right=2cm,top=2cm,bottom=3cm,bindingoffset=0cm]{geometry}
\usepackage{tikz}
\usepackage{graphicx}
\graphicspath{ {./pictures/} }
\usepackage{subfig}
\usepackage{titletoc}
\usepackage{tikz}
\usepackage{pgfplots}
\usepackage{xcolor}
\usepackage{wrapfig}
\usepackage{todonotes}

\newcommand\myworries[1]{\colorbox{red!30}{TODO #1}}
\newcommand\attention[1]{\colorbox{cyan!30}{#1}}

\title{Sem1Synopsis}
\begin{document}

\author{Александр Мишин, Б01-008а}
\date{}
\maketitle\

Так, вспоминаем наше любимое 
\[A^q + Y = C^q\]

Мы сейчас находимся в режиме, что хотим знать, как решать систему таких дифф. уравнений:
\[\frac{d\overrightarrow{u}}{dt} = f (\overrightarrow{u}, t)\]

\attention{Конечная разностная аппроксимация} - то, что было в прошлом году, надо повторить. 

Пусть у нас есть временной отрезок от 0 до t. Вводим сетку с шагом $\tau$ от 0 до t. Сопоставим исходной дифференциальной задаче некую разностную схему\\ \attention{Cемейство линейных многошаговых методов}:
\[\frac{\alpha_k y_{n+k} + ... \alpha_0 y_n}{\tau} = \beta_k f_{n+k} + \beta_0 f_n\]

Примечание - в одномерии u превращается в y. (Разные источники, это неважно)

\attention{Метод Эйлера (запись неявная, одношаговый)}
\[\frac{y_{n+1} - y_n}{h} = f_{n+1}\]

\attention{Метод Эйлера (запись явная, одношаговый)}
\[\frac{y_{n+1} - y_n}{h} = f_{n+1}\]
\[y_{n+1} = y_n + hf_n\]

\attention{Метод Адамса (явный, 3-го порядка)}
\[\frac{y_{n+3} - y_{n+2}}{h} = \frac{23}{12} f_{n+2} - \frac{16}{12} f_{n+1} + \frac{5}{12} f_n\]
Метод явный. Пусть у нас есть точки $y_0$, $y_1$, $y_2$. Как найти $y_3$? Можно применить одношаговый метод! Для того, чтобы посчитать точку $y_3$, хотелось бы вычислить первые точки, используя одношаговые методы. Метод Эйлера - 1-го порядка. Не очень хорошо, что для 3-го порядка мы пользуемся методами 1-го порядка. Как-то не очень :(. А как повышать порядок для одношаговых методов?

\section*{Методы Рунге-Кутты}

\[y_{n+1} = y_L + h \cdot \sum_{i=1}^s b_i f_i\]
Здесь $f_i = f(x_L + c_i h, y_L + h \sum_{j=1}^{s_i} a_{ij} f_j), i = \overline{1, s}$

Можем интерпретировать так:

\begin{bmatrix}
       $с_1$ & $a_{11}$ & $a_{12}$ & ... & $a_{33}$\\[0.3em]
       $с_2$ & $a_{21}$ & $a_{22}$ & ... & $a_{33}$\\[0.3em]
       $с_3$ & $a_{31}$ & $a_{32}$ & ... & $a_{33}$\\[0.3em]
        & $b_{1}$ & $b_{2}$ & ... & $b_{3}$\\[0.3em]
\end{bmatrix}\\

Таблица Бутчера:
\begin{bmatrix}
       $с_1$ & $0$ & $0$ & ... & $0$\\[0.3em]
       $с_2$ & $0$ & $0$ & ... & $0$\\[0.3em]
       $с_3$ & $a_{31}$ & $0$ & ... & $0$\\[0.3em]
        & $b_{1}$ & $b_{2}$ & ... & $b_{3}$\\[0.3em]
\end{bmatrix}\\

\subsection*{Рассмотрим пример:}
\begin{bmatrix}
       0 & $0$ & $0$\\[0.3em]
       $1/2$ & $1/2$ & $0$\\[0.3em]
        & 0 & $1$\\[0.3em]
\end{bmatrix}\\

\[y_{L+1} = y_l + h \cdot f_2\]
\[f_1 = f(x_L, y_L)\]
\[f_2 = f(x_L + \frac{1}{2} h, y_L + h \cdot \frac{1}{2} f_1)\]

Метод Хойна 3-го порядка:
\begin{bmatrix}
       0 & 0 & 0 & 0\\[0.3em]
       1/3 & 1/3 & 0 & 0\\[0.3em]
       2/3 & 0 & 2/3 & 0\\[0.3em]
        & 1/4 & 0 & 3/4 \\[0.3em]
\end{bmatrix}\\

Классическая МРК 4-го порядка:

\begin{bmatrix}
       0 & 0 & 0 & 0 & 0\\[0.3em]
       1/2 & 1/2 & 0 & 0 & 0\\[0.3em]
       1/2 & 0 & 1/2 & 0 & 0\\[0.3em]
       1 & 0 & 1 & 0 & 0\\[0.3em]
        & 1/6 & 2/6 & 1/6 & 1/6 \\[0.3em]
\end{bmatrix}\\

Обычно у нас будет выполняться условие Куты - $c_i = \sum a_{ij}$.
\\
p = 1:
\[\sum_{i = 1}^s b_i = 1\]
p = 2:
\[\sum_{i = 1}^s b_i c_i = 1/2\]
p = 3:
\[\sum_{i=1}^s b_i c_i^2 = 1/3, \sum_{i, j = 1} ^ s b_i a_ij c_j = 1/6\]
p = 4:
\[\sum_{i=1}^s b_i c_i^3 = 1/4, \sum_{i, j= 1} b_i c_i a_ij c_j = 1/8\]
\[\sum b_i a_ij c_j^2 = 1 /12, \sum b_j a_ij a_jk c_k = 1/24\]

\section*{Вывод условия устойчивости системы}
Пишем условие задачи Коши:

\[y'(x) = f(x, y)\]
\[y(x_0) = y_0\]
\[x \in [x_0, x_L]\]

Пусть $\varphi(x)$ удовлетворяет системе Коши. Тогда $\varphi '(x) = f (x', \varphi(x))$
\[y'_x(x) = f(x, y(x)) = f(x, \varphi(x)) + \frac{\partial f(x, \varphi(x))}{\partial y}(y(x) - \varphi(x))\]
\[y'_x - \varphi'_x = \frac{\partial f}{\partial y} (y - \varphi)\]
\[\Delta'_x = \frac{\partial f}{\partial x} \Delta\]
\[\Delta'_x = J(x) \Delta(x)\]
Используем метод Эйлера:
\[\frac{\Delta_{L+1} - \Delta_L}{h} = J(x_L) \Delta_L\]
\[\Delta_{L+1} = (E + hJ(x_L)) \Delta_L\]
\[\Delta_{L+1} = R(z) \Delta_L\]
Здесь $z = Yh$, R(z) - функция устойчивости. В  одномерном случае:
$$|R(z)| = |z + 1|, z \in \mathds{C}$$
Условие устойчивости:
\[|R(z)| < 1\]

\end{document}
