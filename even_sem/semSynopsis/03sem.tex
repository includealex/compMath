\documentclass[a4paper,12pt]{article} 
\usepackage[T2A]{fontenc}			
\usepackage[utf8]{inputenc}			
\usepackage[english,russian]{babel}	
\usepackage{amsmath,amsfonts,amssymb,amsthm,mathrsfs,mathtools} 
\usepackage{cancel}
\usepackage{hhline}
\usepackage{multirow}
\usepackage[colorlinks, linkcolor = purple, citecolor = purple]{hyperref}
\usepackage{upgreek}\usepackage[left=2cm,right=2cm,top=2cm,bottom=3cm,bindingoffset=0cm]{geometry}
\usepackage{tikz}
\usepackage{graphicx}
\graphicspath{ {./pictures/} }
\usepackage{subfig}
\usepackage{titletoc}
\usepackage{tikz}
\usepackage{pgfplots}
\usepackage{xcolor}
\usepackage{wrapfig}
\usepackage{todonotes}

\newcommand\myworries[1]{\colorbox{red!30}{TODO #1}}
\newcommand\attention[1]{\colorbox{cyan!30}{#1}}

\title{Sem3Synopsis}
\begin{document}

\author{Александр Мишин, Б01-008а}
\date{}
\maketitle\

\section*{Пример 1}
\[k_1 = f(x_n + \frac{1}{3}h, y_n + h (\beta_{11} k_1 + \beta_{12} k_2))\]
\[k_2 = f(x_n + h \alpha_2, y_n + h(\beta_{21} k_1 + \frac{1}{4}k_2))\]
\[y_{n+1} = y_n + h (\frac{3}{4}k_1 + c_2 k_2)\]

\begin{bmatrix}
       $1/3$ & $\beta_{11}$ & $\beta{12}$\\[0.3em]
       $\alpha_2$ & $\beta_{21}$ & $1/4$ \\[0.3em]
       & $3/4$ & $c_2$ \\[0.3em]
\end{bmatrix}\\

Условие Кутты:
\[\beta_{11} + \beta_{12} = \frac{1}{3}\]
\[\beta_{21} + \frac{1}{4} = \alpha_2\]
Условие на аппроксимацию:\\
p = 1:
\[\frac{3}{4} + c_2 = 1 \Rightarrow c_2 = \frac{1}{4}\]
p = 2:
\[\sum b_i c_i = \frac{1}{2} = \frac{1}{3}\cdot \frac{3}{4} + \alpha \cdot \frac{1}{4} \Rightarrow \alpha_2 = 1\]
p = 3:
\[\sum b_i c_i^2 = \frac{1}{3} = \frac{3}{4} (\frac{1}{3})^2 + 1^2 \cdot \frac{1}{4}\]
\[\sum b_i a_j c_ij^2 = \frac{1}{6}\]
\[\frac{3}{4}(\frac{1}{3} \beta_{11} + 1 \cdot \beta_{12}) + \frac{1}{4} \cdot (\frac{1}{3}\cdot\frac{3}{4} + 1 \cdot \frac{1}{4})\]


\begin{bmatrix}
       $1/3$ & $5/12$ & $-1/12$\\[0.3em]
       $1$ & $3/4$ & $1/4$ \\[0.3em]
       & $3/4$ & $1/4$ \\[0.3em]
\end{bmatrix}

Мы пришли к конкретному \attention{методу Радо} (неявный метод 3-го порядка).

\section*{Пример 2}
\[R(z) = 1 + z b^T(E - zA) ^ {-1} \overrightarrow{1}\]

A = \begin{bmatrix}
       5/12 & -1/12\\[0.3em]
        3/4 & -1/4\\[0.3em]
\end{bmatrix}\\
E - zA  = \begin{bmatrix}
       $1 - z \cdot \frac{5}{12}$ & $z\cdot 1/12$\\[0.3em]
        $-3/4 z$ & $1 - 1/4 z$\\[0.3em]
\end{bmatrix}\\
$b^T$ = \begin{bmatrix}
    3/4 & 1/4\\[0.3em]
\end{bmatrix}

Считаем R(z), учитывая, что 
$\overrightarrow{1} = $\begin{bmatrix}
    1 \\[0.3em]
    1 \\[0.3em]
\end{bmatrix}

\[R(z) = \frac{2z + 6}{(z-2)^2 + 2}\]
Это и есть функция устойчивости.\\
Проверить на А-устойчивость и L-устойчивость - следующая задача.\\
А-устойчивость:

\[|R(z)| > 1 > \overline{C}\]
R(z) - регулярная функция для  R(z) < 0.

\[|6 + i2y|^2 \leq |6 - y^2 - i4y|^2 \]
\[36 + 4y^2 \leq (6 - y^2)^2 + 16y^2\]
\[\Rightarrow y^4 \geq 0\]
$\Rightarrow$ A - устойчивость.\\
А что по L-устойчивости:\\
A-устойчивость + $\lim_{z \rightarrow + \infty} R(z) = 0$
Всё выполняется $\Rightarrow$ L-устойчива.

\section*{Метод Розенброка}
\[y' = f(y) \text{ - автономный}\]

Допустим, стадии будут выражаться в следующем виде:
\[k_i = h f(y_0 + \sum_{j=1}^{i-1} a_{ij} k_j + a_{ii}k_i\]
\[y_1 = y_0 + \sum_{i=1}^s b_i k_i\]

Применяется одноразово метод Ньютона.
\[k_i = h f(g_i) + h f'(g_i)a_{ii}k_i\]
\[g_i = y_0 + \sum_{j=1}^{i=1} a_{ij}k_j\]

\attention{Метод Розенброка}
\[\overrightarrow{k_i} = h \overrightarrow{f} (\overrightarrow{y_0} + \sum_{j=1}^{i-1} \alpha_{ij}\overrightarrow{k_j}) + h J \sum_{j=1}^i \gamma_{ij} k_j\]
\[\overrightarrow{y_1} = \overrightarrow{y_0} + \sum_{i=1}^s b_i \overrightarrow{k_i}\]

\section*{Пример из статьи}
(Трёхстадийный метод Розенброка).
\[y_{n+1} = y_n + p_1 k_1 + p_2 k_2 + p_3 k_3 + ...\]

\[D_n k_1 = h f(y_n)\]
\[D_n k_2 = h f(y_n + \beta_{21}k_1)\]
\[D_n k_3 = h f(y_n + \beta_{31}k_1 + \beta_{32}k_2)\]

\section*{Линейные многошаговые методы}
\[\alpha_k y_{L+k} + \alpha_{k-1} y_{L+k-1} + ... + \alpha_0 y_L = h (\beta_k f_{L+k} + \beta_{k-1} f_{L+k -1} + ... + \beta_0 f_L \]

Многошаговый метод имееет порядок p
\[\sum_{j=0}^k \alpha_j = 0\]
\[\sum_{j=0}^k \alpha_j j^q = q \sum_{j=0} B_{i}j^{q-1}, q = 1, ..., p\]

\section*{Пример}
\[\alpha_2 u_{n+1} + \alpha_1 u_n + \alpha_0 u_{n-1}\]
Здесь $\alpha_2 = 1, \alpha_1 = -7/8, \alpha_0 = -1/8, b_2 = 5/16, b_1 = 15/16, b_0 - -1/8$

q = 0:
 \[\sum \alpha_j = 0\]
q = 1:
\[\sum \alpha_j j^1 = 1 \cdot \sum_j=0^2 B_j j^{1-1}\]
\[\alpha_0 \cdot 0 + \alpha_1 \cdot 1 + \alpha_2 \cdot 2 = B_0 \cdot (0)^0 + B_1 \cdot 1 + B_2 \cdot 2\]
q = 2 - выполняется. Для q = 3 - нет.


\end{document}

