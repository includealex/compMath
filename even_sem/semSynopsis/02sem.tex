\documentclass[a4paper,12pt]{article} 
\usepackage[T2A]{fontenc}			
\usepackage[utf8]{inputenc}			
\usepackage[english,russian]{babel}	
\usepackage{amsmath,amsfonts,amssymb,amsthm,mathrsfs,mathtools} 
\usepackage{cancel}
\usepackage{hhline}
\usepackage{multirow}
\usepackage[colorlinks, linkcolor = purple, citecolor = purple]{hyperref}
\usepackage{upgreek}\usepackage[left=2cm,right=2cm,top=2cm,bottom=3cm,bindingoffset=0cm]{geometry}
\usepackage{tikz}
\usepackage{graphicx}
\graphicspath{ {./pictures/} }
\usepackage{subfig}
\usepackage{titletoc}
\usepackage{tikz}
\usepackage{pgfplots}
\usepackage{xcolor}
\usepackage{wrapfig}
\usepackage{todonotes}

\newcommand\myworries[1]{\colorbox{red!30}{TODO #1}}
\newcommand\attention[1]{\colorbox{cyan!30}{#1}}

\title{Sem2Synopsis}
\begin{document}

\author{Александр Мишин, Б01-008а}
\date{}
\maketitle\

\section*{Вайбы прошлого семинара}
\[\Delta_{L+1} = R(z) \Delta_L\]
\[R(z) = 1 + z\]
\[|R(z)| < 1\]
\[Z = Y h \]
\[z = \lambda_i h\]
Условие устойчивости:
\[|R(\lambda_i, h) | < 1\]
Следовательно,
\[h < const\]

\section*{Уравнение Далквиста}

\[\frac{du}{dt} = \lambda u = f(t, u) = f(u)\]
Смотрим на $\lambda$ и понимаем, решение у нас затухающее($\lambda < 0$) или возрастающее.
\[u_{L+1} = u_1 + h k_1\]
\[k_1 = f(t, u_1) = f(u_1)\]
\[u_{L+1} = u_L + h \lambda u_2\]
\[u_{L+1} = (1 + \lambda h) u_2\]
Как заместим, устойчивость это $(1 + \lambda h)$.

\subsection*{Пример}
\begin{bmatrix}
       1/2 & 0 & 1/2\\[0.3em]
        & 1 & \\[0.3em]
\end{bmatrix}\\
Какой здесь порядок метода? Второй (смотреть прошлый семинар, как там p определяется). Метод одностадийный, но порядок второй..
\\
Как найти функцию устойчивости?
\[u_{L+1} = u_L + h k_1\]
\[k_1 = f(t_L + \frac{1}{2} h, u_L + h \cdot \frac{1}{2}k_1)\]

Примениим к уравнению Далквиста. Тогда
\[k_1 = \lambda \cdot (u + h \cdot \frac{1}{2} k_1)\]
\[k_1 = \frac{\lambda u_2}{1 - \frac{1}{2} \lambda h}\]
\[\Rightarrow u_{L+1} = u_L + h \frac{\lambda u_2}{1 - \frac{1}{2} \lambda h} = \frac{1 + \frac{1}{2}\lambda h}{1 - \frac{1}{2} \lambda h} \cdot u_L\]
\[|R(z)| = |\frac{1 + z/2}{1 - z/2}| < 1\]
Здесь $z = x + iy$
Тогда 
\[\frac{(1 + x/2)^2 + y^2 / 4}{(1 - x/2)^2 + y^2 / 4} \leq 1\]
\[\Rightarrow 2x \leq 0\]
Значит, устойчивой будет вся левая комплексная плоскость.

\subsection*{Неявный метод Эйлера}
\[u_{L+1} = u_L + h \cdot f(x_{L+1}, y_{L+1})\]
\[k_1 = \frac{y_{L+1} - y_L}{h}\]
\[k_1 = f(x_L + h, y_L + h k_1)\]
Применяем его к уравнению Далквиста. Тогда
\[k_1 = \lambda (y_L + hk_1)\]
\[k_1 = \frac{y_L}{1 - \lambda h}\]
\[u_L + h k_1 = y_L + h \frac{\lambda y_L}{1 - \lambda h} = \frac{1}{1 - \lambda h} y_L\]
\[R(z) = \frac{1}{1 - z}\]
\[|R(z)| \leq 1\]

\section*{Немного определений}
\attention{А-устойчивость}:\\
Метод, имеющий область устойчивости $|R(z)| < 1 \supset \{z: Rez \leq 0\}$ называются А-устойчивыми.\\

\attention{А-$\alpha$ устойчивость}:\\
$\alpha$ - параметр угла раствора, где выполняется условие $Re z \leq 0$.\\

\attention{L-устойчивость}:\\
Метод называется L-устойчивым, если он А-устойчив и выполнено условие, что
\[\lim_{z \xrightarrow{} \infty} R(z) = 0\]

\section*{Функции устойчивости}
Для явных методов, у которых s = p.
\[R(z) = 1 + z + \frac{z^2}{2!} + ... + \frac{z^p}{p!}\]
Неявные методы:
\[R(z) = 1 + z b^T (E-zA)^{-1}\]
(в старых обозначениях таблицы Бутчера).\\

Значит, в большинстве случаев,\\
\[R(z) = \frac{P(z)}{Q(z)}\]
(полиномы степени не выше s). Удобно использовать теорему:\\
\attention{Теорема}\\
Метод Рунге-Кутты с $R(z) = 1 + z b^T (E-zA)^{-1}$ A-устойчив $\Leftrightarrow$
$|R(iy)| \leq 1, \forall y \in R$ и R(z) - аналитиическая функция при Re(z) < 0.

\subsection*{Пример}
\[u' = -400 u\]
\[v' = 3 \cdot 10^{-2} u - 10v + w\]
\[w' = -4\cdot 10^{-2} -12v - 2w\] 
\[u(0) = 1, v(0) = -1, w(0) = 0\]
Матрица Якоби здесь будет матрицей системы. \\

J = \begin{bmatrix}
       -400 & 0 & 0\\[0.3em]
       $3\cdot 10^{-2}$ & $-10$ & $1$\\[0.3em]
        $-4 \cdot 10^{-2}$& -12 & $-2$\\[0.3em]
\end{bmatrix}\\
Ищем собственные значения для этой матрицы. Они равны
\[\lambda_1 = -400, \lambda_2 = -8, \lambda_3 = -4\]

\attention{Жёсткая система}\\
Cистема 
\[\frac{d\overrightarrow{u}}{dt} = f(t, \overrightarrow{u})\]
называется жёсткой, если для всех t, y(t) собственные значения матрицы J(t) удовлетворяют условиям
\[\text{число жёсткости} = \frac{\max |Re \lambda_j|}{\min |Re \lambda_k|} \gg 1\]
к $Re \lambda_j < 0$, $\max |Im \lambda_j| \ll \max |Re \lambda_k|$

\[u_{L+1} = u_L + hk_1\]
\[k_1= f(x_L + \frac{1}{2} h, y_L + \frac{1}{2} hk_1 = F(k_1)= 0\]
Метод Ньютона используем:
\[k_1^{s+1} = k_1^s - \frac{F(k_1^s)}{F'_{k_1}(k_1^s)}\]
\[F'_{k_1}(k_1^s) = \frac{F_{k_1 + 1} - F_{k_1 - 1}}{2h_{k_1}}\]
\end{document}

