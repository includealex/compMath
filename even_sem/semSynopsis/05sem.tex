\documentclass[a4paper,12pt]{article} 
\usepackage[T2A]{fontenc}			
\usepackage[utf8]{inputenc}			
\usepackage[english,russian]{babel}	
\usepackage{amsmath,amsfonts,amssymb,amsthm,mathrsfs,mathtools} 
\usepackage{cancel}
\usepackage{hhline}
\usepackage{multirow}
\usepackage[colorlinks, linkcolor = purple, citecolor = purple]{hyperref}
\usepackage{upgreek}\usepackage[left=2cm,right=2cm,top=2cm,bottom=3cm,bindingoffset=0cm]{geometry}
\usepackage{tikz}
\usepackage{graphicx}
\graphicspath{ {./pictures/} }
\usepackage{subfig}
\usepackage{titletoc}
\usepackage{tikz}
\usepackage{pgfplots}
\usepackage{xcolor}
\usepackage{wrapfig}
\usepackage{todonotes}

\newcommand\myworries[1]{\colorbox{red!30}{TODO #1}}
\newcommand\attention[1]{\colorbox{cyan!30}{#1}}

\title{Sem5Synopsis}
\begin{document}

\author{Александр Мишин, Б01-008а}
\date{}
\maketitle

\section*{Пример}
Пусть есть система \\

\[y'' - \frac{1 + x}{1 + x^2}y' - \cos{x} y = \frac{1}{1 + x^2} , 0 \leq x \leq 1\]

\[2 y'(0) + y(0) = 3\]

\[y'_x(1) = 2\]

Решение

\[h^2 \sim  \frac{y_{n+1} - 2y_N + y_{n-1}} {h^2} - \frac{1 + x_n}{1 + x_n^2} \frac{y_{n+1} - y_{n-1}}{2h} - \cos{x_n y_n} = \frac{1}{1 + x_n^2}, 1 \leq n \leq L - 1\]

\[h \sim 2 \frac{y_1 - y_0}{h} + y_0 = 3, n = 0\]
\[h \sim \frac{y_L - y_{L-1}}{h} = 2, n = L\]

\[[y_1] = [y]_0 + h [y']_0 + \frac{h^2}{2} [y"]_0\]
\[y = \sin{x}\]
\[[y']_0 = \frac{3 - [y]_0}{2}\]
\[[y"]_0 = [y']_0 + \cos{0} [y]_0 + 1\]

Опускаем квадратные скобки, понимая под ними то же самое. (Формализм нам не интересен)

\[y_1 = y_0 + h\frac{3 - y_0}{2} + \frac{h^2}{2} (\frac{3 - y_0}{2} + y_0 + 1), n = 0\]

Итого

\[y_1 = (1 - \frac{1}{2}h + \frac{1}{4}h^2) y" + \frac{5}{4}h^2 + \frac{3}{2}h, n = 0\]

Для правого граничного условия:

\[y_{L-1} = y_L - 2h + \frac{h^2}{2} (\frac{5}{2} + \cos{1} + y_L), n = L\]


Но что будет, если мы будем использовать разностную задачу, а не Тейлора?\\

Со вторым порядком точности:\\

\[y'_0 = \frac{y_1 - y_{-1}}{2h}\]

Расписываем разностное уравнение относительно нулевой точки:\\

\[\frac{y_1 - 2y_0 + y_{-1}}{h^2} - \frac{1 + x_0}{1 + x_0^2} \frac{y_1 - y_{-1}}{h} - \cos{x_0}y_0 = \frac{1}{1 + x_0^2}\]

\[2 \frac{y_1 - y_{-1}}{2h} + y_0 = 3\]

Что такое $y_{-1}$ это \attention{фиктивный узел}. Ну и сам по себе, это - метод фиктивного узла.

\section*{Продолжение}

\[a_n y_{n-1} + b_n y_n + c_n y_{n-1} = d_n\]
\[a_L y_{L-1} + b_L y_L = d_L\]
\[y_0 = \frac{-c_0}{b_0}y_1 + \frac{d_0}{b_0}\]

Переобозначим, теперь\\
\[y_0 = \alpha_1 + \beta_1\]

\[a_1 y_0 + b_1 y_1 + c_1 y_2 = d_1\]
\[(-\alpha_1 y_1 + \beta_1) + b_1 y_1 + c_1 y_2 = d_1\]
\[y_1 = \frac{-c_1}{b_1 - a \alpha_1} y_2 + \frac{d_1 - a_1 \beta_1}{b_1 - a \alpha_1} = \alpha_2 + \beta_2\]

\[y_1 = -\alpha_2 y_2 + \beta_2\]

\[y_i = -\alpha_{i+1} y_{i+1} + \beta_{i+1}\]
\[\alpha_{i+1} = \frac{c_i}{b_i - a_i \alpha_i}\]
\[\beta_{i+1} = \frac{d_1 - a_i \beta_i}{b_i - a_i \alpha_i}\]

\[y_{L-1} = -\alpha_L y_L + \beta_L\]
\[y_{L-1} + b_L y_L = d_L\]
\[a_L (-\alpha_L y_L + \beta_2) + \beta_L y_L = d_L\]
\[y_L = \frac{d_L - a_L \beta_L}{b_L - a_L \alpha_L} = \beta_{L+1}\]
На этом метод прогонки заканчивается.

\end{document}

