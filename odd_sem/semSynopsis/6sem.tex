\documentclass[a4paper,12pt]{article} 
\usepackage[T2A]{fontenc}			
\usepackage[utf8]{inputenc}			
\usepackage[english,russian]{babel}	
\usepackage{amsmath,amsfonts,amssymb,amsthm,mathrsfs,mathtools} 
\usepackage{cancel}
\usepackage{hhline}
\usepackage{multirow}
\usepackage[colorlinks, linkcolor = purple, citecolor = purple]{hyperref}
\usepackage{upgreek}\usepackage[left=2cm,right=2cm,top=2cm,bottom=3cm,bindingoffset=0cm]{geometry}
\usepackage{tikz}
\usepackage{graphicx}
\graphicspath{ {./pictures/} }
\usepackage{subfig}
\usepackage{titletoc}
\usepackage{tikz}
\usepackage{pgfplots}
\usepackage{xcolor}
\usepackage{wrapfig}
\usepackage{todonotes}

\newcommand\myworries[1]{\colorbox{red!30}{TODO #1}}
\newcommand\attention[1]{\colorbox{cyan!30}{#1}}

\title{Sem6Synopsis}
\author{Александр Мишин, Б01-008а}
\date{}
\begin{document}

\maketitle

\section{Метод Ньютона}

\subsection*{\attention{Теорема о сходимости итерационного метода Ньютона}}
Если f(a)f(B) < 0, причём f', f'' непрерывны и знакопостоянны на отрезке локализации [a, b], то если выбрать начальное положение $x_0 \in [a, b] \& f(x_0)f"(x_0) > 0$, то можно вычислить
\[x_{n+1} = x_n - \frac{f(x_n)}{f'(x_n)}\]

Порядок сходимости

\[|x_{n+1} - x^*| \leq \frac{M_2 |x_n - x^*| ^ 2}{2m_1}\]
\[M_2 = max|f''(x)|\]
\[m_a = min|f'(x)|\]

\subsection{Пример}

    \[f(x) = 32x^3 + 20x^2 -11x + 3\]
    Надо найти число корней, локализацию для хотя бы одной точки и $x_n$.
\subsubsection{Шаг 1}
    По теореме Декарта, 
    \[f(-x) = -32x^3 + 20x^2 + 11x + 3\]
    Значит, всего один корень.
\subsubsection{Шаг 2}
    Рассмотрим отрезок локализации f(-1) > 0 $\&$ f(-2) < 0
\subsubsection{Шаг 3}
    Выпишем первые производные
    
    \[f'(x) = 96x^2 + 40x -11\]
    \[f''(x) = 192x + 40\]
    
    По условию теоремы, они должны быть знакопостоянны и непрерывны на отрезке [-2, -1].\\
    Проверяем, удостоверились.\\
    Значит, $x_0 \in [-2, -1]$
    Если бы знак менялся, то нужно было бы отсекать отрезок так, чтобы условия теоремы выполнялись.

\subsection{Модифицированный метод Ньютона}
    \[x_{n+1} = x_n - J^{-1}(x_n) F(x_n)\]


\section{Перекладываем все методы на СНАУ.}
    $$\overrightarrow{F}(\overrightarrow{x}) = 0$$
    Для достаточного условия сходимости метода простой итерации для систем нелинейных уравнений достаточно того, чтобы норма матрицы Якоби $\leq 1$. 
    $$J = ||\frac{\partial \varphi _i}{\partial x_i}||$$

    \[\overrightarrow{x_{n+1}} = \overrightarrow{x_n} - J ^ {-1} (x_n) F(x_n)\]
    \[J = ||\frac{\partial F_i}{\partial x_i}||\]

\subsection{Пример}
    Найти корни системы нелинейных уравнений:
    \[2x_1^2 - x_1 x_2 - 5x_1 + 1 = 0\]
    \[x_1 + 3lgx_1 - x_2^2 = 0\]
    
    \attention{Метод простой итерации}
    Что надо сделать? Прежде всего, надо попытаться найти область локализации. С точки зрения локализации нелинейных уравнений. Можно построить графически. \newpage
    
    \begin{figure}[h!]
        \centering
        \includegraphics[width=12cm]{6SemPic1.png}
        \label{fig:vac}
    \end{figure}

    Видим, что есть две точки пересечения. А как? Ведь значений точек пересечений изначально нет. Выразим $x_1$ и $x_2$.

    \[x_1 = \frac{x_1 (x_2 + 5) - 1}{2}\]
    \[x_2 = \sqrt{x_1 + 3lgx_1}\]

    Мы составляем матрицу Якоби. Берём область локализации корня
    $$G = \{|x_1 - 3,5| \leq 0.1, |x_2 - 2,2| \leq 0.1\}$$

    Работаем в первой норме.
    $$max \{\Sigma|\frac{\partial \varphi_1}{\partial x_i}|, \Sigma|\frac{\partial \varphi_2}{\partial x_i}|\} < 1$$
    Получим оценки для $\frac{\partial \varphi_i}{x_i}$.\\
    Как видим,     
    $$max \{\Sigma|\frac{\partial \varphi_1}{\partial x_i}|, \Sigma|\frac{\partial \varphi_2}{\partial x_i}|\} = 0.54 + 0.27 = 0.81 < 1$$
    Т.о.
    \[||J||_1 = 0.81 < 1\]
    Т.е. достаточные условия выполнены в области локализации.\\
    
    А если я захочу решать \attention{методом Ньютона}?
    \[2x_1^2 - x_1 x_2 - 5x_1 + 1 = 0\]
    \[x_1 + 3lgx_1 - x_2^2 = 0\]
    
    \[\overrightarrow{x_{n+1}} = \overrightarrow{x_n} - J^{-1} \overrightarrow{F}(\overrightarrow{x_n})\]
    Якобиан находим из исходной функции.
    
    
\section{}
\end{document}
