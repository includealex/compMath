\documentclass[a4paper,12pt]{article} 
\usepackage[T2A]{fontenc}			
\usepackage[utf8]{inputenc}			
\usepackage[english,russian]{babel}	
\usepackage{amsmath,amsfonts,amssymb,amsthm,mathrsfs,mathtools} 
\usepackage{cancel}
\usepackage{hhline}
\usepackage{multirow}
\usepackage[colorlinks, linkcolor = purple, citecolor = purple]{hyperref}
\usepackage{upgreek}\usepackage[left=2cm,right=2cm,top=2cm,bottom=3cm,bindingoffset=0cm]{geometry}
\usepackage{tikz}
\usepackage{graphicx}
\graphicspath{ {./pictures/} }
\usepackage{subfig}
\usepackage{titletoc}
\usepackage{tikz}
\usepackage{pgfplots}
\usepackage{xcolor}
\usepackage{wrapfig}
\usepackage{todonotes}

\newcommand\myworries[1]{\colorbox{red!30}{TODO #1}}
\newcommand\attention[1]{\colorbox{cyan!30}{#1}}

\title{Sem5Synopsis}
\author{Александр Мишин, Б01-008а}
\date{}

\begin{document}

\maketitle

\section{Методы решения нелинейных уравнений и систем}

Примеры того, что мы будем решать:
\[ x + \ln{x} = 0\]
Общий вид нелин. уравнения:
\[f(x) = 0\]

Для решения нелинейного уравнения нужно проделать локализацию корней. А затем только - уже применить численный метод на отрезке локализации. Что имеется в виду? Где находятся корни? Можно делать вывод из ОДЗ. Но на практике - либо графический способ, либо аналитический. Например, что мы знаем, если мы взяли [a, b] и знаем, что f(a) * f(b) < 0. \attention{Цель локализации - понять, на каком промежутке искать ответ.}\\

Мы будем искать решение нелинейного уравнения, применив итерацию.
\[x_{n+1} = \varphi(x_n)\]

\section{Принцип сжимающих отображений, базовые понятия}

\attention{Метрическое пространство} - введено понятие расстояния c cоответствующими неравенствами.\\
\attention{Фнудаментальная последовательность} - последовательность называется фундаментальной, если 
\[\forall \varepsilon > 0  \exists h_\varepsilon \forall k \geq \varepsilon \& \forall m \geq n_\varepsilon \longmapsto \rho(x_m, x_n) \leq \varepsilon\]
\attention{Полное метр. пространство} - метрическое пространство называется полным, если всякая фундаментальная последовательность является сходящейся.\\
Оператор А осуществляет \attention{сжимающее отображение} мерт пр-ва К в себя, если $\exists 0 < q < 1$, что $\forall x \& \forall y \in \mathds{R}$ имеет место неравенство 
\[\rho (Ax, Ay) \leq qp(x,y)\]

\subsection*{Метод сжимающих отображений}
    Пусть R -метрическое пространство и оператор А  - осуществляет сж.о. R в себя. Тогда уравнение Ах = х имеет единственное решение, которое может быть получено как предел последовательности $x_{n+1} = Ax_n$.

\subsection*{Достаточное условие cходимости МПИ(НУ)}
\[\rho(x_{n+1}, x) = |x_{n+1} - x^*| = \varphi(x_n) - \varphi(x^*) = |\varphi_x^{'} (\xi)||x_n-x^*| \leq q\]
Вывод - МПИ(НУ) сходится $\forall x_0$, принадлежащей области локализации корня, если $q = |\varphi_x^{'}(\xi)| < 1$, где $\xi$ - любое число из области локализации.\\

\subsection*{Порядок сходимости}
    \[\varphi_x^{'} (x^*) = \varphi_x^{"}(x^*) = ... = \varphi_x^{s} (x^*) = 0\]
    n - порядок сходимости $\varphi^{(n)} (x^*) \neq 0$
    
\subsection*{Условие достижимости точности }
\[|x_n - x^n| \leq \frac{x_{n+1} - x{n}}{1 - q} \leq \frac{q^n}{1 - q} |x_1 - x_0| \leq \varepsilon\]

\subsection*{Пример 1, Полином}
\[f(x) = x ^ 3 - x ^ 2 - 9x + 9 = 0\]
По т.Виетта или \attention{т.Декарта} понимаем, что всего один отрицательный корень.\\
В предположении, что есть один корень ~ 1, мы можем предположить, что область локализации от 0.5 до 1.5. Для отрицательного корня предположим, что область локализации [2, 4]. И ещё одна область [1 , 2].\\
Нам нужно построить какой-нибудь метод для поиска корня.
\[x_{n+1} = \sqrt[3]{x_n^2 + 9x_n - 9}\]
\[x_{n+1} = \frac{x_n^2}{9} - \frac{x_n^2}{9} + 1\]

\[\varphi^{'}(x) = |\frac{x^2}{3} - \frac{2}{9}x| < 1\]
Для отрезка [0.5, 1.5] можно применять.

\[\varphi ^{'}(x) = \frac{1}{3}\frac{2x + 9}{(x^2 + 9x + 9)^2}\]

Можем рассмотреть на отрезке [-4, -2], [2, 4]. Метод пригоден для этих двух участков.\\

\subsection*{Пример 2}
\[x = \ctg{x}\]
известен корень $x^* \approx 0.86$. Есть 4 метода:

\[1) x_{n+1} = \ctg{x_n}\]
\[2) x_{n+1} = \arcctg{x_n}\]\
\[3) x_{n+1} = 2\arcctg{x_n} - x_n\]
\[4) x_{n+1} = \frac{2}{7}x_n + \frac{5}{7}\arcctg{x_n}\]
Ищем производные\\
\[|\varphi_1| = \frac{1}{\sin^2{(x^*)}} > 1\]
\[|\varphi_2| = \frac{1}{1 + (x^*)^2} \approx 0.57 < 1\]
\[|\varphi_3| = |\frac{-2}{1+(x^*)^2} - 1| = |\frac{-(x^*)^2 - 3}{1 + (x^*)^2}| = \frac{3 + (x^*)^2}{1 + (x^*)^2} > 1 \]
\[|\varphi_4| = |\frac{2}{7} - \frac{5}{7}\frac{1}{1+(x^*)^2}| = |\frac{1}{7}\frac{2(x^*)^2 - 3}{1 + (x^*)^2}| \approx 0.12 < 1\]
Подходят 2 и 4, выбираем 4. Потому что чем меньше q, тем лучше сжатие.

\subsection*{Пример 3}
\[x^2 - e^{-x} = 0\]
        \begin{figure}[h!]
            \centering
            \includegraphics[width=12cm]{5SemPic1.png}
            \label{fig:vac}
        \end{figure}
        
        Какой метод берём?
        Из производной функции. А локализацию мы получили из графического метода.
\end{document}
