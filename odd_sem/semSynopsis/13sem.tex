\documentclass[a4paper,12pt]{article} 
\usepackage[T2A]{fontenc}			
\usepackage[utf8]{inputenc}			
\usepackage[english,russian]{babel}	
\usepackage{amsmath,amsfonts,amssymb,amsthm,mathrsfs,mathtools} 
\usepackage{cancel}
\usepackage{hhline}
\usepackage{multirow}
\usepackage[colorlinks, linkcolor = purple, citecolor = purple]{hyperref}
\usepackage{upgreek}\usepackage[left=2cm,right=2cm,top=2cm,bottom=3cm,bindingoffset=0cm]{geometry}
\usepackage{tikz}
\usepackage{graphicx}
\graphicspath{ {./pictures/} }
\usepackage{subfig}
\usepackage{titletoc}
\usepackage{tikz}
\usepackage{pgfplots}
\usepackage{xcolor}
\usepackage{wrapfig}
\usepackage{todonotes}

\newcommand\myworries[1]{\colorbox{red!30}{TODO #1}}
\newcommand\attention[1]{\colorbox{cyan!30}{#1}}

\title{Sem13Synopsis}
\begin{document}

\author{Александр Мишин, Б01-008а}
\date{}
\maketitle\

\section*{ОДУ}

Исходная диффференциальная задача:
\[\frac{du}{dt} = f(t, u)\]

Ставится задача Коши, t > 0, u(0) = 0. Можно решать 
\[\frac{du}{dx} = f(x, u)\]

\[\frac{d^m u}{dt^m} = g(t, u, \frac{du}{dt}, ..., \frac{d^{m-1} u}{d t^{m-1}})\]

Делаем так: $u_1 = u, u_2 = \frac{du_1}{dt}, u_3 = \frac{du_2}{dt}$
\\Тогда 
\[\frac{du^m}{dt} = g(t, u_1, u_2, ..., u_m\]

Исходную дифференциальную задачу ещё записывают так:
\[L(u) = F\]
Здесь\\
L(u) = \begin{cases}
du/dt - f(t, u), t > 0\\
u(0), t = 0
\end{cases}\hskip{;}\hskip{}
F = \begin{cases}
    0, t > 0\\
    $u_0$, t = 0
\end{cases}\\

\section*{Разностная задача}
Разностная задача должна сходиться по сетке 
\[L_\tau (y) = F_\tau\]
Здесь y - сеточное решение(сеточная функция).\\
$U^\tau$ -  проекция точного решения искомой задачи на сетку.

\subsection*{Сходимость разностной задачи к дифференциальной}
\[||y - U^\tau|| \longmapsto 0\]
при $\tau \longmapsto 0$

Пусть сетка [0, 1/3, 2/3]. Тогда проекция нашего решения $u = \sin{t}$ на сетку:\\
\[[u]_0 = 0\]
\[[u]_1 = \sin{1/3} = \sin{\tau}\]
\[[u]_2 = \sin{2/3} = \sin{2\tau}\]

\subsection*{Сходимость 2}
Ребус
\[A^q + Y = C^q\]
Здесь С - сходимость, А - аппроксимация, Y - устойчивость.\\
Разностная задача А исходящую на её решении, если невязка $||r_\tau|| \longmapsto 0$, $\tau \longmapsto 0$, где $r_\tau = L_\tau(U^\tau) - F_\tau$. Если оказалось, что $||r_\tau \leq C_1 \tau ^ q||$, то q - порядок аппроксимации А.\\
Задача разностная устойчива, если две близкие возмущенные задачи одновременно однозначно разрешимы и 
\[L_\tau(y_1) - F_\tau = \xi\]
\[L_\tau(y_2) - F_\tau = \eta\]
Если $||y_1 - y_2|| \leq C_2 (||\xi_\tau|| + ||\eta_\tau||)$

\section*{Классификация}
\[\frac{du}{dt} = f(t, u), t > 0, u(0) = u_0\]
Используем расчётный метод
\[u_{n+1} = u_n + \tau f(t_n, u_n)\]
Это - \attention{явный метод Эйлера первого порядка}.\\\\
\attention{Неявный метод Эйлера первого порядка} :
\[u_{n+1} = u_n + \tau f(t_{n+1}, u_{n+1})\]

\subsection*{Особенности явных и неявных методов по устойчивости}
Все явные методы условно устойчивые!!!\\
Условие устойчивости обычно формулируется на шаг интегрирования.
\[\tau \leq M\]
Н.Я. в большинстве случаев безусловно устойчивые или имеют более широкую область устойчивости, чем аналогичные явные методы.

\section*{Пример 1}
С каким порядком на сетке $D_n = {x_l = lh, l = \overline{0, L}, h = \frac{1}{L}}$ 
\[L_\tau(u) = F_\tau\]
\begin{cases}
\frac{y_{l+1} - 2y_2 + y_{l+1}}{h^2} + 6 \frac{y_{l+1} - y_{l-1}}{2h} + 5y_2 = 0, l = \overline{1, L-1}\\
y_0 = 0\\
y_1 = 2h - 6h^2\\
\frac{d^2 y}{d x^2} + 6 \frac{dy}{dx} + 3y = 0\\
y(0) = 0\\
y'_x(0) = 2\\
\end{cases}\\

Вешаем [].\\
\begin{cases}
\frac{[y]_{l+1} - 2[y]_2 + [y]_{l+1}}{h^2} + 6 \frac{[y]_{l+1} - [y]_{l-1}}{2h} + 5[y]_2 - 0, l = \overline{1, L-1}\\
[y]_0 = 0\\
[y]_1 = 2h - 6h^2\\
\end{cases}\\

Ряд Тейлора:\\
\[[y]_{L \pm 1} = [y]_L \pm h [y'_x]_L + \frac{h^2}{2}[y''_x]_L \pm \frac{h^3}{6}[y'''_x]_L +
\frac{h^4}{24}[y''''_x]_L \pm ...\]

Главные члены погрешности:\\
\[[y''_{xx}]_L + \frac{h^2}{12} [y^{(4)}] + 6\cdot ([y'_x] + \frac{h^2}{6}[y''']) + 5[y]_L = \frac{h^2}{12}[y'''']_L + h^2 [y''']\]

\[[y]_1 = [y]_0 + h[y']_0 + \frac{h^2}{2} [y'']_0 + \frac{h^3}{6}[y''']_0 + \frac{h^4}{24}[y'''']_0\]

\[[y]_0 + h[y']_0 + \frac{h^2}{2} [y'']_0 + \frac{h^3}{6}[y''']_0 + \frac{h^4}{24}[y'''']_0 - 2h - 6h^2\]
\[0 + h[y']_0 + \frac{h^2}{2} [y'']_0 + \frac{h^3}{6}[y''']_0 + \frac{h^4}{24}[y'''']_0 - 2h - 6h^2\]
\[2 + \frac{h^2}{2} [y'']_0 + \frac{h^3}{6}[y''']_0 + \frac{h^4}{24}[y'''']_0 - 2h - 6h^2\]
\[2 + (-12) + \frac{h^3}{6}[y''']_0 + \frac{h^4}{24}[y'''']_0 - 2h - 6h^2\]
Значит, невязка $\leq C_1 h^2$. Значит, итоговый порядок задачи второй. Значит, поставленная задача аппроксимирует исходную задачу с порядком 2.

\end{document}
