\documentclass[a4paper,12pt]{article} 
\usepackage[T2A]{fontenc}			
\usepackage[utf8]{inputenc}			
\usepackage[english,russian]{babel}	
\usepackage{amsmath,amsfonts,amssymb,amsthm,mathrsfs,mathtools} 
\usepackage{cancel}
\usepackage{hhline}
\usepackage{multirow}
\usepackage[colorlinks, linkcolor = purple, citecolor = purple]{hyperref}
\usepackage{upgreek}\usepackage[left=2cm,right=2cm,top=2cm,bottom=3cm,bindingoffset=0cm]{geometry}
\usepackage{tikz}
\usepackage{graphicx}
\usepackage{subfig}
\usepackage{titletoc}
\usepackage{tikz}
\usepackage{pgfplots}
\usepackage{xcolor}
\usepackage{wrapfig}
\usepackage{todonotes}

\newcommand\myworries[1]{\textcolor{red}{TODO #1}}

\title{Sem4Synopsis}
\author{Александр Мишин, Б01-008а}
\date{}

\begin{document}

\maketitle

\section{Ошибка из прошлого семинара}

Было 2 метода -  Метод Якоби и Метод Зейделя.

\subsection{Метод Якоби}
    $$x_{k+1} = D^{-1} (L + U)x_k + D^{-1}b$$
    Теорема [Критерий сходимости метода Якоби:]\\
    det\begin{bmatrix}
       $\lambda$a_{11} & a_{12} & ... & a_{1n} \\[0.3em]
       a_{21} & $\lambda$a_{22} & ... & a_{2n} \\[0.3em]
       ... & ... & ... & ... \\[0.3em]
       a_{n1} & a_{n2} & ... & $\lambda$a_{nn} \\[0.3em]
    \end{bmatrix} = 0, если $|\lambda| < 1$
\subsection{Метод Зейделя}
    $$x_{k+1} = -(L+D)^{-1} Ux_k + (L+D)^{-1}b$$
    Теорема [Критерий сходимости метода Зейделя]\\
    det\begin{bmatrix}
       $\lambda$a_{11} & a_{12} & ... & a_{1n} \\[0.3em]
       $\lambda$a_{21} & $\lambda$a_{22} & ... & a_{2n} \\[0.3em]
       ... & ... & ... & ... \\[0.3em]
       $\lambda$a_{n1} & $\lambda$a_{n2} & ... & $\lambda$a_{nn} \\[0.3em]
    \end{bmatrix} = 0, если $|\lambda| < 1$

\section{Семинар}

    A = \begin{bmatrix}
       \alpha & 0 & \beta \\[0.3em]
        0 & \alpha & 0 \\[0.3em]
        \beta & 0 & \alpha \\[0.3em]
    \end{bmatrix}\\
    Найти критерий сходимости МПИ.\\
    МПИ:\\
    $x_{k+1} = Bx_k + f$ \hspace{1cm} $B = E - A$
    $$det |B - \lambda E| = ((1- \alpha) - \lambda) ^ 3 - \betta((1-\alpha) - \lambda) = 0$$
    Отсюда находим $\lambda_i$.
    
    По методу Якоби:
    det \begin{bmatrix}
       \lambda \alpha & 0 & \beta \\[0.3em]
        0 & \lambda \alpha & 0 \\[0.3em]
        \beta & 0 & \lambda \alpha \\[0.3em]
    \end{bmatrix} = 0\\

    По методу Зейделя:
     det \begin{bmatrix}
       \lambda \alpha & 0 & \beta \\[0.3em]
        0 & \lambda \alpha & 0 \\[0.3em]
        \lambda \beta & 0 & \lambda \alpha \\[0.3em]
    \end{bmatrix} = 0\\

    Существует семейство однопараметрических методов, где вводится парамер $\tau$.
    $$x_{k+1} = (E-\tau A)x_k + \tau f$$
    
    Необходимое и достаточное условие сходимости : $|\lambda(E-\tau A)| < 1 \Rightarrow{} |1-\tau \lambda(A)| < 1$.//
    
    $\tau$ отвечает за скорость сходимости (норма матрицы В, она же норма матрицы итерационного процесса). Существует много частных случаев. Один из таких - если матрица симметрична, то 
    $$0 < \tau < \frac{2}{\lambda_{max}}$$
    $$\tau_\text{оптимальное} = \frac{2}{\lambda_{max} + \lambda_{min}}$$
    
    Задаётся вопрос, а при каких начальных приближениях $x^0$ будет сходиться метод?\\
    A = \begin{bmatrix}
       5 & 4 \\[0.3em]
       1 & 2 \\[0.3em]
    \end{bmatrix} \hspace{1cm}
    f = \begin{bmatrix}
        2 \\[0.3em]
        4 \\[0.3em]
    \end{bmatrix}\\
    
    Находим собственные числа матрицы А: $\lambda_1 = 1,  \hspace{0.3cm} \lambda_2 = 6$.
    Отсюда система уравнений $|1-\tau| < 1, |1-6\tau| < 1 \Rightarrow{} 0 < \tau < \frac{1}{3}$.\\
    
    Для $\tau = 0.1$ сходимость с любого начального приближения. Для $\tau = 2$ - не будет сходиться с любого начального приближения.\\
    
     A = \begin{bmatrix}
       3 & 4 \\[0.3em]
       1 & 3 \\[0.3em]
    \end{bmatrix} \hspace{1cm}
    f = \begin{bmatrix}
        1 \\[0.3em]
        1 \\[0.3em]
    \end{bmatrix}\\
    
    Постройте сходящийся однопараметрический метод простой итерациии.\\
    
    Что будем делать? Можно за два преобразования гарантировать, что метод будет сходящийся.
    Матрица не симметрична - что же делать для ускорения? Привести матрицу А к симметричной. Матрица А хочет стать в виде матрицы. Вот алгоритм для крафта симметричной матрицы, умножая на транспонированную.
    $A^T A \overrightarrow{u} = \overrightarrow{f}$
    $$B \overrightarrow{u} = F'$$
    $$u_{k+1} = (E-\tau B)u_k + \tau F'$$
    $$0 < \tau < \frac{2}{\lambda_{max} (B)} = 0.053 $$
    
    Ок, оно сходится. А как оценить количество итераций, за которое сходится?\\
    
    Ловим аптечку, она же формула: 
    $$N = [\frac{\ln {\varepsilon \frac{1-q}{||F||}}}{\ln q}]_\text{целая часть} + 1$$
    здесь $\varepsilon$ - точность, q - норма матрицы итерационного процесса.\\
    $$u^{(0)} = (0, 0) ^ T$$
    
    У нас F будет зависеть от $\tau$: $F = \tau F'$. Количество итераций будет зависить от $\tau$. Посчитаем для оптимального $\tau$ количество итераций. Напоминание, что мы получили B: \\
    B = \begin{bmatrix}
      25 & 18 \\[0.3em]
      18 & 13 \\[0.3em]
    \end{bmatrix} \\
    
    $$\tau_\text{оптимальное} = \frac{1}{19}$$
    $$F = \tau_\text{оптимальное} A^T$$
    F = $\frac{1}{19}$ \cdot \begin{bmatrix}
      7 \\[0.3em]
      5\\[0.3em]
    \end{bmatrix} \\
    $|E-\tau B|$ = $\frac{1}{19}$\begin{bmatrix}
     -6 & -18 \\[0.3em]
     -18 & 6 \\[0.3em]
    \end{bmatrix} \\
    
    $$||F||_1 = \frac{17}{13}$$
    $$||F||_2 = \frac{24}{19}$$
    Проводим оценку в первой норме:\\
    Достаточное условие сходимости не выполняется для первых двух норм.\\
    
    $$||F||_3 = \frac{\sqrt{74}}{19}$$
    $$||E - \tau B||_3 = max |\lambda_i (E - \tau B)| = \frac{6\sqrt{10}}{19} < 1$$
    
    В итоге если подставить:\\
    $$N = 9155 \text{ итераций.}$$
    
    \subsection{Метод Релаксации}
    
    A = L + D + U
    Здесь в роли параметра $\omega$:\\
    $$(\omega L u^{k+1} + Du^{k+1}) + (\omega - 1)Du^k +\omega U u^k = \omega f$$
    При $\omega = 1$ будет метод Зейделя.
    $$u^{k+1} = -(D + \omega L)^{-1} [(\omega - 1)D + \omega L]u^k + \omega (D + \omega L)f$$
    Известно, что 
    $$0 < \omega_\text{оптимальное} < 2$$
    Для положительно определённой матрицы,
    $$1 < \omega_\text{оптимальное} < 2$$
    - это уже метод верхней релакцсации.
    Метод нижней релаксации:
    $$0 < \omega_\text{оптимальное} < 1$$
    
    Если матрица А оказыавется симметричной, то параметр релаксации будет выражаться через конечное выражение.\\
    
\section{Полезности для домашней лабораторной работы:}
    \subsection*{Степенной метод}
    $$A = A_{n \times n}$$
    $$|\lambda_1| \geq |\lambda_2| \geq |\lambda_n|$$
    $$y^k = Ay^{k-1}$$
    $$\lambda_1 = \frac{y^k}{y^{k-1}}$$
    Критерий останово:
    $$\varepsilon^k = |\lambda_1^{(k)}-\lambda_1^{(k-1)}| \geq \varepsilon$$
    Невязка:
    $$Ax_k = f$$
    $$r_k = ||f-Ax_k||$$
    $$r_k < \varepsilon \text{ - невязка}$$
    
\end{document}

