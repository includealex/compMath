\documentclass[a4paper,12pt]{article} 
\usepackage[T2A]{fontenc}			
\usepackage[utf8]{inputenc}			
\usepackage[english,russian]{babel}	
\usepackage{amsmath,amsfonts,amssymb,amsthm,mathrsfs,mathtools} 
\usepackage{cancel}
\usepackage{hhline}
\usepackage{multirow}
\usepackage[colorlinks, linkcolor = purple, citecolor = purple]{hyperref}
\usepackage{upgreek}\usepackage[left=2cm,right=2cm,top=2cm,bottom=3cm,bindingoffset=0cm]{geometry}
\usepackage{tikz}
\usepackage{graphicx}
\usepackage{subfig}
\usepackage{titletoc}
\usepackage{tikz}
\usepackage{pgfplots}
\usepackage{xcolor}
\usepackage{wrapfig}
\usepackage{todonotes}

\newcommand\myworries[1]{\textcolor{red}{#1}}

\title{Sem2Synopsis}
\author{Александр Мишин, Б01-008а}
\date{}

\begin{document}

\maketitle

\section{Комментарии по д/з}
    Построить график абс. погрешности каждого метода:
    $$f'(x) = \sin(x^2)'$$
    $$f'(x) = \frac{f(x+h) - f(x)}{h}$$
    $$h = \frac{2}{2^n}$$
    В логарифмическом масштабе строим графики $\Delta(h)$. Идейно - должны быть графики прямых.
    $$\Delta f'(h) = |f'_a - f'_\tau|,  n = \overline{1, n}$$
    \\

    Что касается оформления? \\
    Питончик - notebook или collab. Ожидается от нас 
        -ФИО, группа\\
        -Вариант(если задание предусматривает)\\
        -Результаты(графики)\\
        -Код\\

    Нормы в конечномерных пространствах.\\
    
\section{Сам семинар}
    Мы хотим решить $Ax = f$ в большинстве случаев, где А - матрица. А как численно найти x?\\
    Методом Гаусса, Крамора - неинтересно, потому что алг.сложность $n^3$. Может быть ошибка в самом начале, в зависимости от начальных данных. А что такое ошибка при подсчёте
    $\Delta \overline{x}$
    
    Решение - вводить нормы\\
    1)$||x||_1 = \max|x_k| \Leftrightarrow ||A||_1 = \max\limits_{i} \displaystyle\sum_{j=1}^{n} \overline{|a_{ij}|}$\\
    2)$||x||_2 = \Sigma |x_k| \Leftrightarrow ||A||_2 = \max\limits_{j} \displaystyle\sum_{i=1}^{n}
    \overline{|a_{ij}|}$\\
    3)$||x||_3 = \sqrt{(\overline{x}, \overline{x})} \Leftrightarrow ||A||_3 = \sqrt{max \lambda_i (A^*A)}$

    А что есть норма матрицы?\\
    $$||A|| = \sup_{x\neq0}\frac{||Ax||}{||x||}$$
    $$||A||_1 = \sup_{x\neq0}\frac{||Ax||_1}{||x||_1}$$
    $$||\lambda A|| = ||\lambda||||A||$$
    $$||AB|| \leq ||A||||B||$$
    $$||A|| = 0 \Leftrightarrow A = 0$$
    $$||Ax|| \leq ||A||||x||$$
    
    Норма x согласована с нормой А.
    
    $||A\overrightarrow{x}||_1 = |\Sigma a_{ij}x{j}| \leq max(\Sigma |a_{ij}||x_j| \leq (max \Sigma |a_{ij}|) max |x_j| = (max \Sigma |a_{ij}) ||x||_1$
    
    Найдём решение:\\
    $y_k = {sign a_{k1}, ..., sign a_{kn}}$\\
    Пусть при i = k достигается максимум.\\
    Со второй нормой всё то же самое.\\
    А зачем мы всё это делали? Для того, чтобы перейти к разговору про погрешности. Оцениваем погрешность
    
    Теорема
    $$\frac{||\Delta x||}{||x||} \leq \frac{\mu}{1- \mu_0 \frac{||\Delta A||}{||A||}} ( \frac{||\Delta A||}{||A||} + \frac{||\Delta f||}{||f||})$$
    
    Здесь $\mu$ - число обусловленности
    $$\mu = ||A|| \cdot ||A||^{-1}$$
    
    Если $\mu$ находится в пределах от 1 до 100. Если у нас $\mu >> 100$, то задача плохо обусловлена!!! 
    
    В большинстве случаев считаем $\Delta A = 0$. Значит,
    $$\frac{||\Delta x||}{||x||} = \mu \frac{||\Delta f||}{||f||}$$

    Здесь грубая оценка для $\mu$:
    
    $$\mu = ||A||\cdot||A||^{-1}$$

    Точная:
    
    $$\nu = \frac{||\overrightarrow{f}||}{||\overrightarrow{x}||} \cdot ||A||^{-1}$$
    
    Пример, задача. Надо найти $\mu_{1, 2, 3}$ для 
    ||Ax|| = b
    \\
    A = \begin{bmatrix}
       50 & 70  \\[0.3em]
       70 & 101 \\[0.3em]
    \end{bmatrix}
    b = \begin{bmatrix}
       120 \\[0.3em]
       171 \\[0.3em]
    \end{bmatrix}\\
    
    $A^{-1} = \frac{1}{det A}$ \begin{bmatrix}
       50 & 70  \\[0.3em]
       70 & 101 \\[0.3em]
    \end{bmatrix}
    \newpage
    $$||A||_1 \cdot ||A||^{-1}_1 = 171 \frac{171}{150}$$\\
    $$||A||_2 \cdot ||A||^{-1}_2 = \frac{171 \cdot 171}{150}$$\\
    
    Т.к. матрица симметрична, то\\
    $$||A||_3 = \sqrt{\max \lambda(A^T\cdot A)} = \sqrt{\lambda(A^2)} = \max\limits_{i} |\lambda_i (A)|$$
    $$\mu_3 = ||A||_3 \cdot ||A||_3 ^{-1} = \frac{\max \lambda(A)}{\min \lambda(A)} = 150$$
    
\end{document}

