\documentclass[a4paper,12pt]{article} 
\usepackage[T2A]{fontenc}			
\usepackage[utf8]{inputenc}			
\usepackage[english,russian]{babel}	
\usepackage{amsmath,amsfonts,amssymb,amsthm,mathrsfs,mathtools} 
\usepackage{cancel}
\usepackage{hhline}
\usepackage{multirow}
\usepackage[colorlinks, linkcolor = purple, citecolor = purple]{hyperref}
\usepackage{upgreek}\usepackage[left=2cm,right=2cm,top=2cm,bottom=3cm,bindingoffset=0cm]{geometry}
\usepackage{tikz}
\usepackage{graphicx}
\usepackage{subfig}
\usepackage{titletoc}
\usepackage{tikz}
\usepackage{pgfplots}
\usepackage{xcolor}
\usepackage{wrapfig}
\usepackage{todonotes}

\newcommand\myworries[1]{\textcolor{red}{TODO #1}}

\title{Sem3Synopsis}
\author{Александр Мишин, Б01-008а}
\date{}

\begin{document}

\maketitle

\section{Семинар}
    Из прошлого - мы ввели нормы. Ввели - и хорошо.\\
    Сегодня цель - рассмотреть несколько прямых методов.\\
    Что из себя представляет любой прямой метод? - получим ответ.\\
    
\subsection{Прямые(точные) методы для решения СЛАО}
    Компьютер не идеален => компьютер даёт погрешность(из прошлого, к ней - 2 лаба).\\
    Кроме метода Гаусса, в линале мы использовали метод Крамера. Он тоже прямой.\\
    $$Ax = b$$
    
    Давайте посмотрим базовые вещи.\\
    
\subsubsection{Метод Гаусса}
    \begin{bmatrix}
       a_{11} & a_{12} & ... & a_{1n} \\[0.3em]
       a_{21} & a_{22} & ... & a_{2n} \\[0.3em]
       ... & ... & ... & ... \\[0.3em]
       a_{n1} & a_{n2} & ... & a_{nn} \\[0.3em]
    \end{bmatrix}
    \\
    Прямой метод Гаусса:\\
    итерации:\\
    $$\alpha_i^k = \frac{a_{i1} ^ {k-1}}{a_{11} ^ {k-1}}$$
    $$a_{ij}^k = a_{ij} ^ {k-1} - \alpha_i^{k} a_{ij}^{k-1}$$
    $$b_i^k = b_i^{k-1} - \alpha_i^k b_i^{k-1}$$
    $$a_{ij}^k = a_{ij}^{k-1}; b_i^k = b_i^{k-1}; i \leq k; j = \overline{1, n}$$
    
    Также у нас есть обратный ход:
    $$x_n = \frac{b_n^{n-1}}{a_{nn}^{n-1}}$$
    $$x_k = \frac{b_k^{n-1} - \sum_{j=k+1}^n a_{kj}^{n-1} x_j}{a_{kk}^{n-1}}; k = \overline{(n-1), 1}$$
    
\subsubsection{Метод Гаусса с выбором главного элемента:}
    Всё то же самое, что м.Г., НО выбирается $\max |a_{ii}^k|$ по всем элементам в строке
    
    A = \begin{bmatrix}
       2 & -1 & 1  \\[0.3em]
       4 & -2 & 1 \\[0.3em]
       1 & -4 & 6 \\[0.3em]
    \end{bmatrix}
    b = \begin{bmatrix}
       -3 \\[0.3em]
        3 \\[0.3em]
       11 \\[0.3em]
    \end{bmatrix}\\
    \begin{bmatrix}
       2 & -1 & 1  \\[0.3em]
       4 & -2 & 1 \\[0.3em]
       1 & -4 & 6 \\[0.3em]
    \end{bmatrix}
    \xrightarrow{}
    \begin{bmatrix}
       6 & 1 & -4  \\[0.3em]
       1 & 2 & -4 \\[0.3em]
       1 & 4 & 2 \\[0.3em]
    \end{bmatrix}
    \xrightarrow{}
    \begin{bmatrix}
       6 & 1 & -4 & | & 11\\[0.3em]
       0 & \frac{11}{6} & -\frac{10}{3} & | & \frac{-29}{6}\\[0.3em]
       0 &  \frac{23}{6} &  -\frac{4}{9} & | & \frac{7}{4}\\[0.3em]
    \end{bmatrix}
    \xrightarrow{}
    \begin{bmatrix}
       6 & 1 & -4 & | & 11\\[0.3em]
        & \frac{23}{6} & \frac{4}{3} & | & \frac{7}{6}\\[0.3em]
        &  \frac{11}{6} &  \frac{10}{3} & | & \frac{-29}{6}\\[0.3em]
    \end{bmatrix}\\
    
    $$6x_3 + x_1 -4x_2 = 11$$
    $$\frac{23}{6}x_1 -\frac{4}{3}x_2 = \frac{7}{6}$$
    $$372x_2 = 744$$
    
    Отсюда обратный метод и находим $x_2 =2, x_1 = 1, x_3 = 3$\\
    
    Вспомнили, прослезились.
\subsubsection{LU метод}    
    А теперь - хотим сделать так:
    $$Ax = b \xrightarrow{} LUx = b$$
    L = \begin{bmatrix}
       1 & 0 & ... & 0 \\[0.3em]
       l_{21} & 1 & ... &0 \\[0.3em]
       ... & ... & ... & ... \\[0.3em]
       l_{n1} & l_{n2} & ... & 1 \\[0.3em]
    \end{bmatrix}
    ;
    U = \begin{bmatrix}
       d_{11} & d_{12} & ... & d_{1n} \\[0.3em]
       0 & d_{22} & ... & d_{2n} \\[0.3em]
       ... & ... & ... & ... \\[0.3em]
       0 & 0 & ... & d_{nn} \\[0.3em]
    \end{bmatrix}\\
    
    A = \begin{bmatrix}
       10 & -7 & 0 \\[0.3em]
       -3 & 6 & 2 \\[0.3em]
       5 & -1 & 5 \\[0.3em]
    \end{bmatrix};
    L = \begin{bmatrix}
       1 & 0 & 0 \\[0.3em]
       l_{21} & 1 & 0 \\[0.3em]
       l_{31} & l_{32} & 1 \\[0.3em]
    \end{bmatrix};
    U = \begin{bmatrix}
       10 & -7 & 0 \\[0.3em]
       0 & d_{22} & d_{23} \\[0.3em]
       0 & 0 & d_{33} \\[0.3em]
    \end{bmatrix}\\
    \\
    Из этих матриц и предыдущих высказываний находим все $l_{ij}, d_{ij}$.
    $$d_{1i} = a_{1i}$$
    $$d_{22} = \frac{39}{10}$$
    $$d_{23} = 2$$
    $$l_{21} = -0.3$$
    $$l_{31} = 0.5$$
    $$l_{32} = \frac{25}{39}$$
    $$d_{33} = ...$$

    Чтобы разложение было возможно, надо проверять, что все главные миноры матрицы А отличны от нуля. Не забываем, что чаще всего подбираются удачные матрицы. \\
    
\subsubsection{Метод квадратного корня}
    $Ax = b$. Предположим, что А - симметричная. Тогда мы можем $A = LL^T$.\\
    L = \begin{bmatrix}
       l_{11} & 0 & ... & 0 \\[0.3em]
       l_{21} & l_{22} & ... &0 \\[0.3em]
       ... & ... & ... & ... \\[0.3em]
       l_{n1} & l_{n2} & ... & l_{nn} \\[0.3em]
    \end{bmatrix}\\

    $$l_{11} = \sqrt{a_{11}}$$
    $$l_{ik} = \frac{a_{ik} - \sum_{j=1}^{k-1}l_{ik}l_{kj}}{l_{ii}}$$

    \begin{bmatrix}
       6.25 & -1 & 0.5 & | & 7.5  \\[0.3em]
       -1 & 5 & 2.12 & | & -8.68\\[0.3em]
       0.5 & 2.12 & 3.6 & | & -0.24\\[0.3em]
    \end{bmatrix}

    

    $$l_{11} = \sqrt{6.25} = 2.5$$
    $$l_{22} = \frac{a_{21}}{l_{11}} = \frac{-1}{2.5} = -0.4$$
    $$l_{31} = \frac{0.5}{2.5} = 0.2$$
    $$l_{22} = \sqrt{5 - (-0.4)^2} = 2.2$$
    $$l_{32} = ... = 1$$
    $$l_{33} = ... = 1.6$$
    $$L^T x = y ; Ly = b$$
    Получаем решение (0.8, -2, 1)

\subsection{Итерационный метод}
    Итерационные методы для решения СЛАУ.\\
\subsubsection{Метод простой итерации}
    $Ax = b \xrightarrow{} Ax + x = b + x \xrightarrow{} x = x - Ax + b \xrightarrow{} x_k = (E-A)x_{k-1} + b$\\
    $Ax = b \xrightarrow{} x_k = Bx_{k-1} + F$\\

    Теорема[Достаточное условие сходимости:]\\
    Если ||В||= q < 1, то существует единственное решение $x^*$ уравнения $x = Bx + F$ при любомначальном приближении $x_0$ И.П. сходится, причём 
    $$||x^{(k)} - x^*|| \leq q^k ||x^0 - x^*||$$
    Похоже на геом. прогрессии.
    
    Теорема[Критерий сходимости МПИ]\\
    Необходимо и достаточно $|\lambda(B)| < 1$
    
\subsubsection{Однопараметрический метод простой итерации}
    $$x_k = (E-\tau A) x_{k-1} + \tau b$$

\subsubsection{Метод Якоби}
    $$A = L + D + U$$
    L - нижняя треугольная, D - диагональная, U - верхняя треугольная.
    $$Lx + Dx + Ux = b \xrightarrow{} Lx_k + Dx_{k+1} + Ux_k = b$$
    $$x_{k+1} = D^{-1} (L + U)x_k + D^{-1}b$$
    
    Теорема [Критерий сходимости метода Якоби:]\\
    det\begin{bmatrix}
       $\lambda$a_{11} & a_{12} & ... & a_{1n} \\[0.3em]
       a_{21} & $\lambda$a_{22} & ... & a_{2n} \\[0.3em]
       ... & ... & ... & ... \\[0.3em]
       a_{n1} & a_{n2} & ... & $\lambda$a_{nn} \\[0.3em]
    \end{bmatrix} = 0, если $|\lambda| < 1$

\subsubsection{Метод Зейделя}
    $$A = L + D + U$$
    L - нижняя треугольная, D - диагональная, U - верхняя треугольная.
    $$Lx + Dx + Ux = b \xrightarrow{} Lx_{k+1} + Dx_{k+1} + Ux_k = b$$
    $$x_{k+1} = -(L+D)^{-1} Ux_k + (L+D)^{-1}b$$
    
    Теорема [Критерий сходимости метода Зейделя]\\
    det\begin{bmatrix}
       $\lambda$a_{11} & a_{12} & ... & a_{1n} \\[0.3em]
       $\lambda$a_{21} & $\lambda$a_{22} & ... & a_{2n} \\[0.3em]
       ... & ... & ... & ... \\[0.3em]
       $\lambda$a_{n1} & $\lambda$a_{n2} & ... & $\lambda$a_{nn} \\[0.3em]
    \end{bmatrix} = 0, если $|\lambda| < 1$
\end{document}
